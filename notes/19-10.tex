\documentclass[../main.tex]{subfiles}
\graphicspath{{\subfix{../images/}}}
\begin{document}

\begin{claim}
    $G=([n], E_M)$ multigraf je souvislý $\Leftrightarrow h(L_G) = n-1$ 
\end{claim}


\begin{definition}
    $L_G^{(i)} \coloneq $ matice získaná z $L_G$ odstraněním $i$-tého řádku a sloupce.
\end{definition}

\begin{claim}
    $L_G^{(i)} $není$ L_{G-i}$
\end{claim}

\begin{theorem}
    $\forall i \in [n], \det L_G^{(i)} = $ počet koster multigrafu $G = [n, E_M]$
\end{theorem}

\subsection{Eulerovský tah}

\begin{definition}
    Multigraf $G=(V,E_M)$ lze nakreslit jedním tahem $\Leftrightarrow$ $\exists$ tah v $G$ obsahující všechny prvky $E_M$    
\end{definition}


\begin{definition}[přijemnější verze]
    Multigraf $G=(V,E_M)$ lze nakreslit jedním tahem $\Leftrightarrow$ $\exists$ uzavřený tah    
\end{definition}

\begin{claim}
    Nutná podmínka $\exists E$ (Eulerovského tahu) $\implies$ stupeň je sudý a $G$ je souvislý. 
\end{claim}

\begin{theorem}[Eulerova]
    Nutná podmínka je i postačující!
\end{theorem}

\begin{proof}
    Buď $T$ nejdelší (vůči počtu hran) tah v $G$.

    Tvrdíme, že T musí být uzavřený tah. Pro spor 
    předpokládejme, že není. Pak by koncový vrchol T je koncem 
    lichého počtu hran T, avšak $deg_G$ je sudý $\implies \exists f \notin T$ a $T+f$ je delší tah.
    
    T obsahuje všechny hrany

    Máme tah $T=v_0 e_1 v_1 ... e_l v_l$\& $\exists f \in E_M\setminus T$. 
    Protože graf je souvislý tak $f$ souvisí s tahem, pokuď bychom tah začali ve vrcholu který s ním souvisí, tak by byla delší cesta
\end{proof}


\begin{corollary}
    $G$ je souvislý a počet vrcholů lichého stupně je buď 0 nebo 2 $\Leftrightarrow \exists $ tah v $G$ obsahující všechny hrany. 
\end{corollary}

\begin{proof}
    Spojíme hranou 2 vrcholy lichého stupně, z Eulerovy věty vytvoříme uzavřený tah, a přidaný vrchol odstraníme.
\end{proof}


\end{document}