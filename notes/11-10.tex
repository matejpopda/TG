\documentclass[../main.tex]{subfiles}
\graphicspath{{\subfix{../images/}}}
\begin{document}


\section{Algoritmy na grafech}

\begin{remark}
    Souvislý $G=(V,E)$ a $T=(V,F)$ kostra $G$, pak platí
    $\forall e \in E\setminus F: T + e$ obsahuje $!1$ cyklus $C_e^T$
\end{remark}

\begin{definition}
    $G=(V,E)$ graf, $e\in\binom{V}{2}\setminus E, G+e$ značí graf
    $(V, E\cup \{ e\})$
\end{definition}

\begin{definition}
    Cyklu $C_e^T$ se říká fundamentální cyklus $e$ v $G$
    vůči $T$.
\end{definition}

\begin{remark}
    Souvislý $G=(V,E)$, $T=(V,F)$ kostra, $e\in E\setminus F$.
    $\forall f \in E(C_e^T)$ platí, že $(T+e)-f$ je kostra $G$.
\end{remark}


\begin{definition}
    Vážený graf $G_W = (V,E,w)$ tak ,že  $(V,E)$ je graf, $w: E\mapsto \mathbb{R}$
\end{definition}

\subsection{Nejmenší kostra/ Minimum Spanning Tree/ MST}
Pro daný vážený graf (souvislý), najdi kostru $G$ nejmenší váhy.

"SMŮNO" (s malou úhonou na obecnosti) předpokládejme že $w$ je prostá.

\begin{claim}
    Souvislý graf $G= (V,E)$ a $w: E\mapsto \mathbb{R}$ prostá.
    Potom $(V,E,w)$ má právě jednu MST. 
\end{claim}

\subsubsection*{Kruskalův algoritmus}
Vstup $(V,E,w)$, $|E|=m, |V|=n$
\begin{enumerate}
    \item Seřaď $E$ dle vah. $e_1, e_2,...,e_m$ tak, že $w(e_i) < w(e_{i+1}), \forall i$
    \item $T^\emptyset \coloneq \emptyset$
    \item Pro $i=1$ do $m$. Pokud $T^{i-1} + e_i$ cyklus? Pokud ano přeskočíme,  jinak hranu přidáme.
\end{enumerate}


\subsection{Hledání nejkratší cesty}
$w$ nemusí být prostá ale musí být nezáporná. 
\begin{theorem}[Dijkstra]
    $G=(V,E,w)$ vážený graf, $w: E\mapsto \mathbb{R}_0^+, s\in V$
    potom existuje acycklický faktor $T_s=(V,F)$
    tak , že  $\forall v \in V, dist_{G}(s,v) = dist_{T_S}(s,v)$    
\end{theorem}

\begin{proof}[Dijsktrův algoritmus]
    \begin{enumerate}
        \item $T^{(0)} \coloneq (\{ s\}, \emptyset), i=0$
        \item Dokud $\exists e \in E$ napříč $V(T^{(i)})$ a $V \setminus V(T^{(i)})$
        \begin{enumerate}
            \item $e_i  = \{x,y\} \coloneq$ zvol tu hranu tak, že $x\in V(T^{(i)})$ a $y\notin V(T^{(i)})$ a výraz $dist_{T^{(i)}} (s,x) + w(e_i)$ je nejmenší množina
            \item $T^{(i)} \coloneq (V(T^{(i)})) \cup \{ y \}, E(T^{(i)}) \cup \{ e_i \}$
            \item $i\coloneq i + 1$ 
        \end{enumerate}
        \item $T_S \coloneq (V, {e_0,..., e_{i-1}})$
    \end{enumerate}
     
\end{proof}




\end{document}