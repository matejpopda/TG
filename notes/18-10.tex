\documentclass[../main.tex]{subfiles}
\graphicspath{{\subfix{../images/}}}
\begin{document}


\begin{definition}
    Multigraf $G=(V,E_M)$ kde $E_M$ je multimnožina (element tam  může být i víckrát) obsahující prvky $\binom{V}{2}$
\end{definition}

Alternativně 
\begin{definition}
    Vážený graf $(V,E,w)$, kde $w$ je násobnost, $w:E\mapsto \mathbb{N}$.
\end{definition}

Mezi těmito grafy existuje 1:1 korespondence.

\begin{remark}
    V literatuře někdy mohou existovat existovat 
    smyčky (hrana z vrcholu do sebe sama) to my neuvažujeme. 
\end{remark}


\begin{definition}
    Multigraf $G=(V,E,w)$ je souvislý, pokud je graf $(V,E)$ souvislý.
\end{definition}


\begin{definition}
    Pro multigraf $G=(V,E_m)$ definujme počet koster $T(G) \coloneq \# F \subseteq E_m$ tak, že $F$ je strom. 
\end{definition}


\begin{definition}
    Kontrakce hrany (pro multigrafy! bez smyček)
    
    Značíme $G/e$, multigraf po zkontrahování $e\in G$.

    Výsledkem je multigraf $(V\setminus\{x,y\}\cup \{z\}, E^\star_M)$, kde
    \begin{enumerate}
        \item $\{u,v\}\in E_M$ tak, že $\{u,v\} \cap \{x,y\} = \emptyset \Leftrightarrow \{u,v\} \in E^\star_M$
        \item $\{u,x\}\in E_M$ tak, že $u\neq y \Leftrightarrow \{u,z\}\in E^\star_M$
        \item $\{u,y\}\in E_M$ tak, že $u\neq x \Leftrightarrow \{u,z\}\in E^\star_M$ 
    \end{enumerate}
    včetně násobnosti. 
\end{definition}


\begin{claim}
    $T(G) = \#$ koster G obsahující $e + \#$ koster G neobsahujících e $= T(G/e) + T(G-e)$. 
\end{claim}

\begin{definition}
    Matice sousednosti (adjacency matrix)

    Multigraf $G = ([n], E_M)$

    Matice $A_G\in \mathbb{N}^{nxn}$, symetrická, na diagonále nuly. $a_{ij} = $ násobnost hrany $\{i,j\}$
\end{definition}

\begin{claim}
    $G=([n], E_M)$ multigraf, $A_G$ jeho matice sousednosti 
    $\forall k \in \mathbb{N}_{>0}, i,j\in V$ platí, že $(A_G)^k$ 
    je matice $nxn$ kde na pozici $i,j$ je počet sledů s $k$ hranami 
    v $G$ z $i$ do $j$.
\end{claim}


\begin{definition}
    Matice incidence grafu $G=([n], E_M), |E_M| = m$ je booleovská matice.

    Matice  o rozměru $nxm$.

    Sloupec odpovídá hraně $e_i$, řádek obsahuje 1 pokud hrana obsahuje vrchol $j$. Jinak 0.

    Značí se jako $B_G$
\end{definition}


\begin{claim}
    $G=(V,E)$ graf, $\lambda_1$ největší vlastní číslo jeho spektra (tj. matice $A_G$). $\delta$ je minimální stupeň vrcholů, $\Delta$ je maximální stupeň vrcholů.

    Pak platí $\delta \leq \text{průměrný stupeň} \leq \lambda_1 \leq \Delta$
\end{claim}


\begin{definition}
    Laplaceova matice multigrafu $G=([n], E_M)$.

    Symetrická matice $L_G\in \mathbb{N}^{nxn}$. Na diagonále má stupně vrcholů, $l_{ij} = -$ násobnosti hrany $\{i,j\}$
\end{definition}

\begin{claim}
    $L_G$ je singulární
\end{claim}

\begin{claim}
    $L_G\geq 0$, tj, pozitivně semi-definitní.
\end{claim}



\end{document}