\documentclass[../main.tex]{subfiles}
\graphicspath{{\subfix{../images/}}}
\begin{document}

\section*{3. domácí úlohy}

\subsection*{1a)}
\subsubsection*{Zadání}
Buď $G=(V,E)$ $k$-regulární bipartitní graf s částmi $L$ a $R$. Dokažte, že $|L|=|R|$.

\subsubsection*{Řešení}


\subsection*{1b)}
\subsubsection*{Zadání}
Nalezněte, až na isomorfismus, všechny bipartitní grafy $G$, jejichž doplněk $\bar{G}$ je též bipartitní.

\subsubsection*{Řešení}


\subsection*{2)}
\subsubsection*{Zadání}
Pro graf $G=(V,E)$ označme průměrem $G$ maximální délku nejkratší cesty mezi nějakými dvěma vrcholy, tzn. $\max_{u,v\in V} \text{dist}_{G}(u,v)$. Dokažte, že má-li graf průměr alespoň 4, tak jeho doplněk má průměr nejvýše 2.

\subsubsection*{Řešení}


\subsection*{3)}
\subsubsection*{Zadání}
Pro graf $G=(V,E)$ označme $g(G)$ délku nejkratšího cyklu, který $G$ obsahuje; pokud je $G$ acycklický, definujeme $g(G)\coloneq \infty$.
Poznamenejme, že parametru $g(G)$ se také říká obvod grafu.

Buď $d\geq 3$ a $r\geq 2$ pevné a $G=(V,E)$ libovolný $d$-regulární graf. Dokažte
\begin{enumerate}
    \item Je-li $g(G)=2r$, tak  potom $|V| \geq \frac{2(d-1)^r -2}{d-2}$
    \item Je-li $g(G) = 2r + 1$ tak potom $|V|\geq \frac{d(d-1)^r -2}{d-2}$
\end{enumerate}

\subsubsection*{Řešení}


\subsection*{4)}
Připomeňme, že pro graf $G=(V,E)$ nazveme hranu $e\in E$ mostem, jestliže podgraf $(V,E\setminus\{e\})$ má více komponent souvislosti než $G$.\\
Analogicky nazveme vrchol $v\in V$ artikulací, jestliže indukovaný podgraf množinou $V\setminus{v}$ má více komponent souvislosti než $G$.

\subsection*{4a)}
\subsubsection*{Zadání}
Dokažte, že každý $2k$- regulární graf neobsahuje most.



\subsubsection*{Řešení}


\subsection*{4b)}
\subsubsection*{Zadání}
Dokažte, že každý $k$-regulární bipartitní graf neobsahuje artikulaci.

\subsubsection*{Řešení}


\subsection*{5a)}
\subsubsection*{Zadání}
Buď $G=(V,E)$ neprázdný graf a buď $d \coloneq \frac{|E|}{|V|}$. Dokažte, že $G$ obsahuje indukovaný podgraf $H$ splňující $\delta(H) > d$.

(Nápověda: zkuste z $G$ postupně odebírat vrcholy stupně nejvýše $d$.)

\subsubsection*{Řešení}


\subsection*{5b)}
\subsubsection*{Zadání}
Pro každé přirozené $d\geq 1$ a každé $\varepsilon\in(0,1)$ zkonstruujte graf $G_{d,\varepsilon}= (V,E)$
takový, že splňuje $\frac{|E|}{|V|} > d - \varepsilon$ a zároveň $\delta(H)\leq d$ pro každý podgraf $H \subseteq G_{d,\varepsilon}$.


\subsubsection*{Řešení}

\end{document}