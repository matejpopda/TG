\documentclass[../main.tex]{subfiles}
\graphicspath{{\subfix{../images/}}}

\setcounter{secnumdepth}{0}
\begin{document}

\section{4. domácí úlohy}

Na tomto úkolu jsem pracoval spolu s Janem Plecháčkem.

\subsection{1a)}
\subsubsection*{Zadání}
Buď $G=(V,E)$ graf s $n$ vrcholy a spektrem $\lambda_1 \geq \lambda_2 \geq \dots \geq \lambda_n$. 
Dokažte, že\begin{equation*}
    \sum_{i\in[n]} \lambda_i^2 = 2|E|
\end{equation*}


\subsubsection*{Řešení}

Využijeme dvě věty z lineární algebry. 

\begin{lemma*}
    Nechť má matice $\mathbb{A}$ vlastní číslo $\lambda$, poté má matice $\mathbb{A}^2$ vlastní číslo $\lambda^2$
\end{lemma*}
\begin{proof}
    Víme, že platí $\mathbb{A}x = \lambda x$. Pak jednoduše máme 
    \begin{equation*}
        \mathbb{A}^2x = \mathbb{A}(\mathbb{A}x) = \mathbb{A}(\lambda x) = \lambda \mathbb{A} x = \lambda^2 x
    \end{equation*}
\end{proof}

\begin{lemma*}
    Stopa matice $\mathbb{A}$ je rovna součtu vlastních čísel matice (včetně násobnosti).

    \textit{Bez důkazu.}
\end{lemma*}


Graf má matici sousednosti $A_G$, její prvky značme $a_{ii}$. Pak za pomoci předchozích dvou vět platí 
\begin{equation*}
    \sum_{i\in[n]} \lambda_i^2 = tr(A_G^2)
\end{equation*} 


Označme prvky matice $A_G^2$ jako $b_{ii}$. Nyní stačí explicitně spočítat stopu. Máme
\begin{equation*}
    \sum_{i\in[n]} \lambda_i^2 = tr(A_G^2) = \sum_{i=1}^n b_{ii} = \sum_{i=1}^n \sum_{k=1}^n a_{ik}a_{ki} = \sum_{i=1}^n \sum_{k=1}^n a_{ik}
\end{equation*} 

V posledním kroku jsme využili vlastnosti, že matice $A_G$ je symetrická a navíc toho, že její hodnoty jsou buď 1 nebo 0. 
Poslední člen je navíc zřejmě roven $2|E|$, protože počítáme počet jedniček v matici sousednosti, a každá hrana spojuje právě dva vrcholy.  

Celkem tedy dostáváme hledané 
\begin{equation*}
    \sum_{i\in[n]} \lambda_i^2 = 2|E|
\end{equation*}
\qed 


\subsection{1b)}
\subsubsection*{Zadání}

Buď $G$ souvislý graf s maximálním stupněm $D$, a nechť spektrum $G$ má největší vlastní číslo $\lambda_1$. Dokažte, že $D=\lambda_1$ právě když $G$ je $D$-regulární.  

\subsubsection*{Řešení}

Dokažme $\impliedby$: 

Plyne z tvrzení z přednášky $D = \delta \leq \lambda_1 \leq \Delta = D$. Z toho zřejmě $D = \lambda_1$.\\
Dokažme $\implies$: 

Platí $D=\lambda_1$, vezměme libovolný vlastní vektor (označme $x$) tohoto vlastního čísla. Označme $x_k$ v absolutní hodnotě jeho největší prvek.
Vezměme vektor $y = \frac{1}{x_k}x$ (vlastní vektor $D$) a jeho největší hodnota je 1.

Vezměme $k$-tý řádek matice $A_G$ a označme ho jako $a_k$

Platí, že $A_G y = D y$ a také platí $a_K^T y = D y$. Z tohoto dostáváme následující sadu nerovností.
\begin{equation*}
    D = D \cdot y_k = a_k^T y = \sum_{i=1}^n a_{k,i}y_i \leq \sum_{i=1}^n a_{k,i} y_k = \sum_{i=1}^n a_{k,i} = deg(k) \leq D 
\end{equation*}

Proto u nerovnosti $\sum_{i=1}^n a_{k,i}y_i \leq \sum_{i=1}^n a_{k,i} y_k$ vždy nastává rovnost. Protože  $y_j \leq 1, \forall j\in[n]$ dostáváme $y_k = y_j = 1$.

Pokud v nerovnosti použijeme místo $k$ řádek $j$ (kde $j$ je soused $k$ v grafu) dostaneme $deg(j) = D$. Opakováním procesu dospějeme k tomu, že $y_j = 1$ a $deg(j) = D$ protože graf je souvislý.



\subsection{2)}
\subsubsection*{Zadání}
Buď $G=(V,E)$ graf s $n$ vrcholy a spektrem $\lambda_1 \geq \lambda_2 \geq \dots \geq \lambda_n$.
Dokažte, že graf je bipartitní právě tehdy když $\lambda_i = - \lambda_{n-i+1}$ pro každé $i\in [n]$

\subsubsection*{Řešení}


Dokažme $\impliedby$: 

Nechť $\lambda_1 \geq \lambda_2 \geq \dots \geq \lambda_n$ a platí $\lambda_i = - \lambda_{n-i+1}$. Ze symetrie $A_G$ víme, že je diagonalizovatelná, a pro $k$-tou mocninu pak platí, 
$A_G^k = X D^k X^{-1}$. Vlastní čísla matice $A_G^k$ jsou tedy $\lambda_i^k$.

Z přednášky víme, že $(A^k_G)_{ij}$ je počet sledů délky $k$ z vrcholu $i$ do $j$. Chci ukázat, že graf nemá cyklus liché délky, tedy chci ukázat, že $(A^k_G)_{ii} = 0$ pro každé liché $k_3$.

Protože máme $\lambda_i = - \lambda_{n-i+1}$, tak také platí $\lambda_i^k = (- \lambda_{n-i+1})^k$ a proto tedy $Tr(A_G^K) = \sum_{i=1}^n \lambda_i^k = 0$ pro každé liché $k$.

Protože, ale jsou všechny prvky matice nezáporné, tak platí, že $(A^k_G)_{ii} \geq 0$, protože je ale stopa matice rovna nule, tak musí platit $(A^k_G)_{ii} = 0$.

Graf tedy neobsahuje cyklus liché délky a proto je bipartitní.\\
Dokažme $\implies$: 

Nechť $\lambda_1 \geq \lambda_2 \geq \dots \geq \lambda_n$ a $G$ bipartitní. Označme vrcholy v množině $L$ čísly od $1$ do $i$, v $R$ od $i+1$ do $n$.
Matice sousednosti pak vypadá následovně.
\begin{equation*}
    A_G = \begin{pmatrix}
        0 & B \\
        B^T & 0 
        \end{pmatrix} 
\end{equation*} 
 To plyne z toho, že vrcholy z $L$ a $R$ nejsou spojeny. 
 
 Nechť $\vec{x}$ je libovolný vlastní vektor $A_G$ k nějakému vlastnímu číslu $\lambda$, označme prvních $i$ prvků jako $\vec{y}$ a zbylé prvky jako $\vec{z}$.Pak máme 

 \begin{equation*}
    A_G \vec{x} = \begin{pmatrix}
        0 & B \\
        B^T & 0 
        \end{pmatrix} \begin{pmatrix}
            \vec{y}  \\
            \vec{z}  
            \end{pmatrix} = \lambda\begin{pmatrix}
                \vec{y}  \\
                \vec{z}  
                \end{pmatrix} = \lambda \vec{x}
 \end{equation*}

 Z prostřední rovnosti víme, že $B^T \vec{y} = \lambda \vec{z}$ a $B \vec{z} = \lambda \vec{y}$. Pak je ale i vektor $\vec{w}= \begin{pmatrix}
    \vec{y}  \\
     - \vec{z}  
    \end{pmatrix}$ vlastním vektorem, protože

    \begin{equation*}
        A_G \vec{w} = \vec{x} = \begin{pmatrix}
            0 & B \\
            B^T & 0 
            \end{pmatrix} \begin{pmatrix}
                \vec{y}  \\
                - \vec{z}  
                \end{pmatrix} = \begin{pmatrix}
                    -B \vec{z}  \\
                    B^T \vec{y}  
                    \end{pmatrix} = \begin{pmatrix}
                        -\lambda \vec{y}  \\
                        \lambda \vec{z}  
                        \end{pmatrix} = - \lambda \begin{pmatrix}
                             \vec{y}  \\
                            - \vec{z}  
                            \end{pmatrix} = -\lambda \vec{w}
    \end{equation*}
    Z toho dostáváme, že pokud je $\lambda$ vlastní číslo, tak pak je i $-\lambda$ vlastní číslo.

    Protože $A_G$ je symetrická tak víme, že je diagonalizovatelná. To znamená, že pro všechny vlastní čísla se rovná geometrická a algebraická násobnost. Z tohoto a předchozího pozorování pak plyne, že má-li vlastní číslo
    $\lambda_k$ algebraickou násobnost $i$, tak i vlastní číslo $-\lambda_k$ bude mít násobnost $i$. To plyne z toho, že $\lambda_k$ má $i$ nezávislých vlastních vektorů, a  ty můžeme podle předchozího upravit na $i$ vlastních vektorů $-\lambda_k$.

    Když seřadíme vlastní velikosti, tak již dostaneme námi hledané tvrzení, tj.  $\lambda_i = - \lambda_{n-i+1}$ pro každé $i\in [n]$



\end{document}