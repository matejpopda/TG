\documentclass[../main.tex]{subfiles}
\graphicspath{{\subfix{../images/}}}
\begin{document}


\section*{1. domácí úlohy}

\subsection*{1)}
\subsubsection*{Zadání}
Graf nazvěme \textit{asymetrickým} obsahuje-li jeho grupa automorfismů pouze identitu.
Zkonstruujte nekonečně mnoho navzájem neisomorfních asymetrických grafů.
\subsubsection*{Řešení}
Jako naši množinu neisomorfních asymetrických grafů vezměme\\ množinu $\mathcal{G} = \{G_7, G_8, ...\}$.\\
$G_n$ definujeme jako
\begin{equation*}
    G_n = ([n], \{\{ 1,2 \}, \{ 1,3\}, \{ 3,4 \}, \{ 1,5 \}, \{ 5,6 \}, \{ 6,7 \}, ..., \{ n-1, n \} \} )
\end{equation*}

Protože se jedná a grafy s různým počtem vrcholů neexistuje mezi nimi isomorfní zobrazení. 

Dokažme, že se jedná o asymetrické grafy. \todo{here}



\subsection*{2)}
\subsubsection*{Zadání}
Pro daný graf $G=(V,E)$, uvažujme následující relaci $\leadsto$ na $V$: pro dva (ne nutně různé) vrcholy $u,v\in V$ řekneme, že 
$u$ je v relaci s $w$, tzn. $u\leadsto w$, jestliže existuje cesta v $G$ z $u$ do $w$. Dokažte, že $\leadsto$ je ekvivalence na $V$
\subsubsection*{Řešení}
Aby relace byla ekvivalencí, musí být reflexivní, symetrická a tranzitivní.

Předpokládáme tedy graf $G=(V,E)$, $v_i\in V$, $e_j\in E$, kde $i,j \in [n+k]$.

Dokažme reflexivitu, tj. $\forall u\in V: u\leadsto u$.\\
Toto tvrzení platí triviálně, jako cestu volíme jednočlennou posloupnost $(u)$.

Dokažme symetrii, tj. $\forall u,w\in V: u\leadsto w \implies w\leadsto u$.\\
Předpokládáme tedy existenci cesty z $u$ do $w$, tedy posloupnost 
\begin{equation*}
    (u, e_1, v_1, e_2, ..., v_{n-1}, e_{n-1}, w)
\end{equation*}
Zřejmě existuje cesta z $w$ do $u$, stačí přetočit předchozí posloupnost, tedy
\begin{equation*}
    (w, e_{n-1}, v_{n-1}, ..., v_1, e_1, w)
\end{equation*}

Nakonec dokažme tranzitivitu, tj. $\forall u,w,x\in V: u\leadsto w \wedge w\leadsto x \implies u\leadsto x$\\
Existuje tedy cesta z $u$ do $w$ a cesta z $w$ do $x$, tedy
\begin{equation*}
    (u, e_1, v_1, e_2, ..., v_{n-1}, e_{n-1}, w)
\end{equation*}
\begin{equation*}
    (w, e_{n+1}, v_{n+1}, e_{n+2}, ..., v_{n+k-1}, e_{n+k-1}, x)
\end{equation*}
Nejprve řešme triviální případ, kdy $\nexists a \in [n], b \in \{n+1, n+2, ..., n+k\}, v_a = v_b$ .\\
V tomto případě existuje cesta ve tvaru
\begin{equation*}
    (u, e_1, v_1, e_2, ..., v_{n-1}, e_{n-1}, w, e_{n+1}, v_{n+1}, e_{n+2}, ..., v_{n+k-1}, e_{n+k-1}, x)
\end{equation*} 
Nakonec řešme případ kdy $\exists a \in [n], b \in \{n+1, n+2, ..., n+k\}, v_a = v_b$.\\
Nejmenší $a$ pro které $\exists b$ tak, že rovnost $v_a = v_b$ platí označme jako $h$, korespondující $b$ (existuje pouze jedno) označme jako $g$.
Jako cestu můžeme poté volit jako
\begin{equation*}
    (u, e_1, v_1, e_2, ..., v_{j-1}, e_{j-1}, v_h = v_g , e_{g+1}, v_{g+1},..., v_{n+k-1}, e_{n+k-1}, x)
\end{equation*} 

Tímto jsme ukázali, že $\leadsto$ je ekvivalence.  
\qed


\subsection*{3a)}
\subsubsection*{Zadání}
Dokažte, že každý graf $G$ s alespoň dvěma vrcholy obsahuje dvojici různých vrcholů se stejným stupňem. Jinými slovy, $\exists u,w \in V(G)$
tak, že $u\neq w$ a $deg_G(u) = deg_G(w)$.
\subsubsection*{Řešení}

Ukážeme sporem. \\Předpokládáme, že existuje graf $G = (V,E)$ ve kterém neexistuje dvojice různých vrcholů se stejným stupňem.\\
Jako $V$ vezměme $[n]$\\
Budeme využívat vztah 
\begin{equation}\label{hw1-3-handshake}
    \sum_{v\in V} deg_G (v) = 2|E|
\end{equation}

% Protože vrcholy mají různé stupně tak můžeme levou stranu omezit zdola jako sumu $\sum_{j=0}^{n-1} j$. Z toho plyne, že
% \begin{equation*}
%     \frac{(n-1)n}{2} \leq 2|E| = 2 \binom{n}{2} = 2 \frac{n!}{2! (n-2)!} = n(n-1)
% \end{equation*}
% Nalezněme pro které $n$ tato nerovnost neplatí.
% \begin{align*}
%     (n-1)n &\leq 2 n(n-1)\\
%     1 &\leq 2 
% \end{align*} 

Odhadněme levou stranu zhora, jako $\sum_{v\in V} |V|$, tj. každý je soused s každým.\\
Tento odhad ještě upravme tak, abychom využili předpokladu, že neexistuje dvojice vrcholů se stejným stupněm.\\
\begin{equation*}
    \sum_{v\in V} |V| \geq \sum_{v\in V} |V| - v =  \sum_{j=0}^{n-1} n - j = n^2 - \frac{(n-1)n}{2} = \frac{n(n+1)}{2}
\end{equation*}

Pravou stranu \eqref{hw1-3-handshake} můžeme přepsat jako 
\begin{equation*}
    2|E| = 2 \binom{n}{2} = 2 \frac{n!}{2! (n-2)!} = n(n-1)
\end{equation*}

Tu stejnou sumu omezme zdola jako
\begin{equation*}
    \frac{(n-1)n}{2}
\end{equation*}

Nakonec tedy dostáváme 
\begin{equation*}
    \frac{n(n+1)}{2} \geq 2|E| \geq \frac{(n-1)n}{2}
\end{equation*}
\begin{equation*}
    n+1 \geq (n-1)
\end{equation*}
\todo{dodelat}


\subsection*{3b)}
\subsubsection*{Zadání}
Buď $\delta$ přirozené číslo. Dokažte, že každý nenulový graf, kde každý vrchol má stupeň alespoň $\delta$, obsahuje cestu s $\delta$ hranami.
\subsubsection*{Řešení}


\subsection*{4)}
\subsubsection*{Zadání}
Nazvěme graf $G$ \textit{doplňkem sebe sama} platí-li, že $G$ je isomorfní svému doplňku $\bar{G}$. 
Zkonstruujte nekonečně mnoho navzájem neisomorfních grafů $G$ jež jsou doplňkem sebe sama.
\subsubsection*{Řešení}



\subsection*{5)}
\subsubsection*{Zadání}
Graf nazvěme \textit{d-regulárním}, jestliže všechny jeho vrcholy mají stupeň přesně $d$. 
Určete všechny dvojice čísel $n$ a $d$, kde $0\leq d\leq n-1$, takové, že existuje $d$-regulární graf s $n$ vrcholy.
\subsubsection*{Řešení}



\end{document}