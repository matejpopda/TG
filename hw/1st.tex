\documentclass[../main.tex]{subfiles}
\graphicspath{{\subfix{../images/}}}
\begin{document}


\section*{1. domácí úlohy}

\subsection*{1)}
\subsubsection*{Zadání}
Graf nazvěme \textit{asymetrickým} obsahuje-li jeho grupa automorfismů pouze identitu.
Zkonstruujte nekonečně mnoho navzájem neisomorfních aysmetrických grafů.
\subsubsection*{Řešení}


\subsection*{2)}
\subsubsection*{Zadání}
Pro daný graf $G=(V,E)$, uvažujme následující relaci $\leadsto$ na $V$: pro dva (ne nutně různé) vrcholy $u,v\in V$ řekneme, že 
$u$ je v relaci s $w$, tzn. $u\leadsto w$, jestliže existuje cesta v $G$ z $u$ do $w$. Dokažte, že $\leadsto$ je ekvivalence na $V$
\subsubsection*{Řešení}
Aby relace byla ekvivalencí, musí být reflexivní, symetrická a tranzitivní.

Předpokládáme tedy graf $G=(V,E)$, $v_i\in V$, $e_j\in E$, kde $i,j \in [n+k]$.

Dokažme reflexivitu, tj. $\forall u\in V: u\leadsto u$.\\
Toto tvrzení platí triviálně, jako cestu volíme jednočlennou posloupnost $(u)$.

Dokažme symetrii, tj. $\forall u,w\in V: u\leadsto w \implies w\leadsto u$.\\
Předpokládáme tedy existenci cesty z $u$ do $w$, tedy posloupnost 
\begin{equation*}
    (u, e_1, v_1, e_2, ..., v_{n-1}, e_{n-1}, w)
\end{equation*}
Zřejmě existuje cesta z $w$ do $u$, stačí přetočit předchozí posloupnost, tedy
\begin{equation*}
    (w, e_{n-1}, v_{n-1}, ..., v_1, e_1, w)
\end{equation*}

Nakonec dokažme tranzitivitu, tj. $\forall u,w,x\in V: u\leadsto w \wedge w\leadsto x \implies u\leadsto x$\\
Existuje tedy cesta z $u$ do $w$ a cesta z $w$ do $x$, tedy
\begin{equation*}
    (u, e_1, v_1, e_2, ..., v_{n-1}, e_{n-1}, w)
\end{equation*}
\begin{equation*}
    (w, e_{n+1}, v_{n+1}, e_{n+2}, ..., v_{n+k-1}, e_{n+k-1}, x)
\end{equation*}
Nejprve řešme triviální případ, kdy $\nexists a \in [n], b \in \{n+1, n+2, ..., n+k\}, v_a = v_b$ .\\
V tomto případě existuje cesta ve tvaru
\begin{equation*}
    (u, e_1, v_1, e_2, ..., v_{n-1}, e_{n-1}, w, e_{n+1}, v_{n+1}, e_{n+2}, ..., v_{n+k-1}, e_{n+k-1}, x)
\end{equation*} 
Nakonec řešme případ kdy $\exists a \in [n], b \in \{n+1, n+2, ..., n+k\}, v_a = v_b$.\\
Nejmenší $a$ pro které $\exists b$ tak, že rovnost $v_a = v_b$ platí označme jako $h$, korespondující $b$ (existuje pouze jedno) označme jako $g$.
Jako cestu můžeme poté volit jako
\begin{equation*}
    (u, e_1, v_1, e_2, ..., v_{j-1}, e_{j-1}, v_h = v_g , e_{g+1}, v_{g+1},..., v_{n+k-1}, e_{n+k-1}, x)
\end{equation*} 

Tímto jsme ukázali, že $\leadsto$ je ekvivalence.  
\qed


\subsection*{3a)}
\subsubsection*{Zadání}
Dokažte, že každý graf $G$ s alespoň dvěma vrcholy obsahuje dvojici různých vrcholů se stejným stupňem. Jinými slovy, $\exists u,w \in V(G)$
tak, že $u\neq w$ a $deg_G(u) = deg_G(w)$.
\subsubsection*{Řešení}


\subsection*{3b)}
\subsubsection*{Zadání}
Buď $\delta$ přirozené číslo. Dokažte, že každý nenulový graf, kde každý vrchol má stupeň alespoň $\delta$, obsahuje cestu s $\delta$ hranami.
\subsubsection*{Řešení}


\subsection*{4)}
\subsubsection*{Zadání}
Nazvěme graf $G$ \textit{doplňkem sebe sama} platí-li, že $G$ je isomorfní svému doplňku $\bar{G}$. 
Zkonstruujte nekonečně mnoho navzájem neisomorfních grafů $G$ jež jsou doplňkem sebe sama.
\subsubsection*{Řešení}



\subsection*{5)}
\subsubsection*{Zadání}
Graf nazvěme \textit{d-regulárním}, jestliže všechny jeho vrcholy mají stupeň přesně $d$. 
Určete všechny dvojice čísel $n$ a $d$, kde $0\leq d\leq n-1$, takové, že existuje $d$-regulární graf s $n$ vrcholy.
\subsubsection*{Řešení}



\end{document}