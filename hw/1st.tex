\documentclass[../main.tex]{subfiles}
\graphicspath{{\subfix{../images/}}}
\begin{document}


\section*{1. domácí úlohy}

\subsection*{1)}
Graf nazvěme \textit{asymetrickým} obsahuje-li jeho grupa automorfismů pouze identitu.
Zkonstruujte nekonečně mnoho navzájem neisomorfních aysmetrických grafů.

\subsection*{2)}
Pro daný graf $G=(V,E)$, uvažujme následující relaci $\leadsto$ na $V$: pro dva (ne nutně různé) vrcholy $u,v\in V$ řekneme, že 
$u$ je v relaci s $w$, tzn. $u\leadsto w$, jestliže existuje cesta v $G$ z $u$ do $w$. Dokažte, že $\leadsto$ je ekvivalence na $V$



\subsection*{3a)}
Dokažte, že každý graf $G$ s alespoň dvěma vrcholy obsahuje dvojici různých vrcholů se stejným stupňem. Jinými slovy, $\exists u,w \in V(G)$
tak, že $u\neq w$ a $deg_G(u) = deg_G(w)$.



\subsection*{3b)}
Buď $\delta$ přirozené číslo. Dokažte, že každý nenulový graf, kde každý vrchol má stupeň alespoň $\delta$, obsahuje cestu s $\delta$ hranami.


\subsection*{4)}
Nazvěme graf $G$ \textit{doplňkem sebe sama} platí-li, že $G$ je isomorfní svému doplňku $\bar{G}$. 
Zkonstruujte nekonečně mnoho navzájem neisomorfních grafů $G$ jež jsou doplňkem sebe sama.



\subsection*{5)}
Graf nazvěme \textit{d-regulárním}, jestliže všechny jeho vrcholy mají stupeň přesně $d$. 
Určete všechny dvojice čísel $n$ a $d$, kde $0\leq d\leq n-1$, takové, že existuje $d$-regulární graf s $n$ vrcholy.



\end{document}