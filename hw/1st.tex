\documentclass[../main.tex]{subfiles}
\graphicspath{{\subfix{../images/}}}

\setcounter{secnumdepth}{0}
\begin{document}


\section{1. domácí úlohy}

\subsection{1)}
\subsubsection*{Zadání}
Graf nazvěme \textit{asymetrickým} obsahuje-li jeho grupa automorfismů pouze identitu.
Zkonstruujte nekonečně mnoho navzájem neisomorfních asymetrických grafů.
\subsubsection*{Řešení}
Jako naši množinu neisomorfních asymetrických grafů vezměme\\ množinu $\mathcal{G} = \{G_7, G_8, ...\}$.\\
$G_n$ definujeme jako
\begin{equation*}
    G_n = ([n], \{\{ 1,2 \}, \{ 1,3\}, \{ 3,4 \}, \{ 1,5 \}, \{ 5,6 \}, \{ 6,7 \}, ..., \{ n-1, n \} \} )
\end{equation*}

Protože se jedná a grafy s různým počtem vrcholů neexistuje mezi nimi isomorfní zobrazení. 

Ukažme, že se jedná o asymetrické grafy.\\
Aby zobrazení $f$ bylo automorfismus, tak musí platit $f(1) = 1$, protože se jedná o a jediný uzel stupně 3.\\
Dále určitě musí platit $f(2) = 2$, protože to je jediný uzel stupně 1, který je zároveň spojený s uzlem stupně 3.\\
Nakonec musí platit, že $f(3) = 3$ a $f(4) = 4$, protože uzel 4 musí sousedit s uzlem, který sousedí s uzlem stupně 3.\\
Podobné tvrzení tedy platí i pro uzly $5, ..., n$. Tedy existuje pouze jedno zobrazení a to identita. 




\subsection{2)}
\subsubsection*{Zadání}
Pro daný graf $G=(V,E)$, uvažujme následující relaci $\leadsto$ na $V$: pro dva (ne nutně různé) vrcholy $u,v\in V$ řekneme, že 
$u$ je v relaci s $w$, tzn. $u\leadsto w$, jestliže existuje cesta v $G$ z $u$ do $w$. Dokažte, že $\leadsto$ je ekvivalence na $V$
\subsubsection*{Řešení}
Aby relace byla ekvivalencí, musí být reflexivní, symetrická a tranzitivní.

Předpokládáme tedy graf $G=(V,E)$, $v_i\in V$, $e_j\in E$, kde $i,j \in [n+k]$.

Dokažme reflexivitu, tj. $\forall u\in V: u\leadsto u$.\\
Toto tvrzení platí triviálně, jako cestu volíme jednočlennou posloupnost $(u)$.

Dokažme symetrii, tj. $\forall u,w\in V: u\leadsto w \implies w\leadsto u$.\\
Předpokládáme tedy existenci cesty z $u$ do $w$, tedy posloupnost 
\begin{equation*}
    (u, e_1, v_1, e_2, ..., v_{n-1}, e_{n-1}, w)
\end{equation*}
Zřejmě existuje cesta z $w$ do $u$, stačí přetočit předchozí posloupnost, tedy
\begin{equation*}
    (w, e_{n-1}, v_{n-1}, ..., v_1, e_1, w)
\end{equation*}

Nakonec dokažme tranzitivitu, tj. $\forall u,w,x\in V: u\leadsto w \wedge w\leadsto x \implies u\leadsto x$\\
Existuje tedy cesta z $u$ do $w$ a cesta z $w$ do $x$, tedy
\begin{equation*}
    (u, e_1, v_1, e_2, ..., v_{n-1}, e_{n-1}, w)
\end{equation*}
\begin{equation*}
    (w, e_{n+1}, v_{n+1}, e_{n+2}, ..., v_{n+k-1}, e_{n+k-1}, x)
\end{equation*}
Nejprve řešme triviální případ, kdy $\nexists a \in [n], b \in \{n+1, n+2, ..., n+k\}, v_a = v_b$ .\\
V tomto případě existuje cesta ve tvaru
\begin{equation*}
    (u, e_1, v_1, e_2, ..., v_{n-1}, e_{n-1}, w, e_{n+1}, v_{n+1}, e_{n+2}, ..., v_{n+k-1}, e_{n+k-1}, x)
\end{equation*} 
Nakonec řešme případ kdy $\exists a \in [n], b \in \{n+1, n+2, ..., n+k\}, v_a = v_b$.\\
Nejmenší $a$ pro které $\exists b$ tak, že rovnost $v_a = v_b$ platí označme jako $h$, korespondující $b$ (existuje pouze jedno) označme jako $g$.
Jako cestu můžeme poté volit jako
\begin{equation*}
    (u, e_1, v_1, e_2, ..., v_{j-1}, e_{j-1}, v_h = v_g , e_{g+1}, v_{g+1},..., v_{n+k-1}, e_{n+k-1}, x)
\end{equation*} 

Tímto jsme ukázali, že $\leadsto$ je ekvivalence.  
\qed


\subsection{3a)}
\subsubsection*{Zadání}
Dokažte, že každý graf $G$ s alespoň dvěma vrcholy obsahuje dvojici různých vrcholů se stejným stupňem. Jinými slovy, $\exists u,w \in V(G)$
tak, že $u\neq w$ a $deg_G(u) = deg_G(w)$.
\subsubsection*{Řešení}

Ukážeme sporem. \\
Jako $V$ vezměme $[n]$.\\
Předpokládáme, že existuje graf $G = (V,E)$ ve kterém neexistuje dvojice různých vrcholů se stejným stupňem.

Proto tedy můžeme vrcholy označit tak, že $deg_G(v_0) = 0, deg_G(v_1) = 1, ..., deg_G(v_n)=n$.\\
Ale toto je spor, protože vrchol $v_0$ nemá být spojen s žádným dalším vrcholem ale, vrchol $v_n$ má být spojen se všemi vrcholy.

Tím jsme ukázali, že každý graf obsahuje dvojici různých vrcholů se stejným stupněm. 


\subsection{3b)}
\subsubsection*{Zadání}
Buď $\delta$ přirozené číslo. Dokažte, že každý nenulový graf, kde každý vrchol má stupeň alespoň $\delta$, obsahuje cestu s $\delta$ hranami.
\subsubsection*{Řešení}

Důkaz provedeme sestrojením cesty v obecném nenulovém grafu $G=([n], E)$, kde každý vrchol má stupeň alespoň $\delta$.


Počátek cesty označme jako $v_0$. 
Zvolme jednu hranu z $E$ (je jich alespoň $\delta$) které obsahují $v_0$, tu označme jako $e_1$.
Druhý vrchol, který $e_1$ obsahuje, označme jako $v_1$.\\
Počet hran z $E$ obsahujících $v_1$, které ještě součástí hledané cesty nejsou, je alespoň $\delta-1$.
Jednu z nich zvolme a označme $e_2$, vrchol který v hledané cestě ještě není označme $v_2$.

Tento krok opakujeme, v $k$-tém kroku máme cestu $v_0, e_1, v_1, ..., v_k$. \\
Počet hran z $E$, které obsahují $v_k$ a neobsahují žádný z vrcholů $v_j, j\in[k-1]$ je $\delta-k$.\\
Jednu z těchto hran zvolíme a označíme $e_{k+1}$.\\
Vrchol který v hledané cestě ještě není označme $v_{k+1}$.\\


Tento krok můžeme opakovat dokud $k \neq \delta$. Poté není zaručena existence hrany z $E$, 
která by spojovala $v_\delta$ s vrcholem, který v námi konstruované cestě není.

V grafu tedy existuje cesta $v_0, e_1, v_1, ...,e_\delta , v_\delta$, která zřejmě obsahuje $\delta$ hran.
\qed


\subsection{4)}
\subsubsection*{Zadání}
Nazvěme graf $G$ \textit{doplňkem sebe sama} platí-li, že $G$ je isomorfní svému doplňku $\bar{G}$. 
Zkonstruujte nekonečně mnoho navzájem neisomorfních grafů $G$ jež jsou doplňkem sebe sama.
\subsubsection*{Řešení}

Označme $H_n$ graf s $n$ vrcholy, kde každý vrchol je propojený s každým, tedy $(n-1)$-regulární graf. 
Jeho doplněk označme jako $\bar{H}_n$.

Sestrojme graf jako na obrázku.

\begin{center}
    \centering
    \includegraphics*[width=0.3\linewidth]{images/hw1-4a.png}
\end{center}

Čarou mezi dvěma grafy rozumíme to, že každý vrchol z jednoho grafu je spojený s každým vrcholem v druhém grafu. 

Pokuď provedeme doplněk tohoto grafu, dostaneme graf jak je vidět na následujícím obrázku. Je zřejmé, že podgraf $H_n$ se stane svým doplňkem a že pokuď mezi dvěma podgrafy jsou všechny vrcholy spojeny, tak po provedení doplňku nebudou spojeny žádné.

\begin{center}
    \centering
    \includegraphics*[width=0.5\linewidth]{images/hw1-4.png}
\end{center}

Levý graf je tedy doplněk pravého, a tyto grafy jsou isomorfní. Graf je tedy námi hledaným doplňkem sebe sama. 

Grafy pro různá $n$ jsou mezi sebou zřejmě neisomorfní, protože mají různý počet vrcholů. 

Tím jsme zkonstruovali nekonečně mnoho navzájem neisomorfních grafů $G$ jež jsou doplňkem sebe sama.




\subsection{5)}
\subsubsection*{Zadání}
Graf nazvěme \textit{d-regulárním}, jestliže všechny jeho vrcholy mají stupeň přesně $d$. 
Určete všechny dvojice čísel $n$ a $d$, kde $0\leq d\leq n-1$, takové, že existuje $d$-regulární graf s $n$ vrcholy.
\subsubsection*{Řešení}
Předpokládejme graf s $n$ vrcholy, $\mathbb{V} = [n]$.

Zaveďme následující značení pro $k>0$
\begin{equation*}
    E_k = \left\{ \{ 0, k\bmod n \}, \{ 1, 1 + k\bmod n \}, \{ 2, 2+k\bmod n \}, ..., \{ n - 1, k - 1\bmod n \}\right\}
\end{equation*}
Všimněme si, že každé číslo se objeví právě ve dvou neuspořádaných dvojící.

Pro $k=0$ definujme
\begin{equation*}
    E_0 = \{\}
\end{equation*}

Vidíme, že triviálně platí následující tvrzení $\forall a,b <n/2 \in \mathbb{N}_0, a\neq b \implies E_a \cap E_b = \emptyset$.



Nyní můžeme explicitně konstruovat $d$-regulární grafy následujícím způsobem. 

$d$-regulární graf označme jako $G_d$, vezměme $2h=n-1$ pro $n$ liché, $2h=n-2$ pro $n$ sudé. 

\begin{align*}
    G_0 &= (\mathbb{V}, E_0 )\\
    G_2 &= (\mathbb{V}, E_0 \cup E_1)\\
    G_4 &= (\mathbb{V}, E_0 \cup E_1 \cup E_2)\\
    G_6 &= (\mathbb{V}, E_0 \cup E_1 \cup E_2 \cup E_3)\\
    \dots&\dots\dots\\
    G_{2h-2} &= (\mathbb{V}, \bigcup_{i=0}^{h-1} E_i)\\
    G_{2h} &= (\mathbb{V}, \bigcup_{i=0}^{h} E_i)
\end{align*}


Tímto způsobem jsme zkonstruovali všechny sudé regulární grafy pro libovolný počet vrcholů.


Dále využijme následujícího tvrzení.
\begin{lemma*}
    Pokud pro $n$ vrcholů existuje $d$-regulární graf, tak pro $n$ vrcholů existuje i $(n-1-d)$-regulární graf.    
\end{lemma*}
\begin{proof}
    Důkaz je triviální, stačí si uvědomit že v $d$-regulárním grafu s $n$ vrcholy, má každý vrchol $d$ hran z $n-1$ možných. \\
    Jeho doplněk je poté $(n-1-d)$-regulární graf. 
\end{proof}

Díky tomuto lematu a předchozí konstrukci dokážeme pro každé sudé $n$ zkonstruovat všechny $d$-regulární grafy, kde platí $0\leq d\leq n-1$. 

Nakonec se zabývejme zbývajícím případem, kdy $n$ a $d$ jsou lichá čísla. Vyjděme z principu sudosti a $d$-regularity hledaného grafu.
\begin{align*}
    \sum_{v\in V} deg_{G}(V) &= 2|E|\\
    \sum_{v\in V} d &= 2|E|\\
    n d &= 2|E|
\end{align*}
Jak vidíme tak pro $n$ a $d$ liché tato rovnost není splněna, protože na levé straně je liché a na pravé straně sudé číslo.
A tedy neexistuje $d$-regulární graf s $n$ vrcholy, pro $n$ a $d$ lichá.

Množina všech uspořádaných dvojic $(n,d)$ tak, že pro ně existuje $d$-regulární graf s $n$ vrcholy nakonec vypadá následovně.
\begin{equation*}
    \left\{  (n,d)\in\mathbb{N}^2 \mid  0\leq d\leq n-1 \wedge nd \text{ je sudé číslo} \right\}
\end{equation*}


\end{document}