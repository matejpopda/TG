\documentclass[../main.tex]{subfiles}
\graphicspath{{\subfix{../images/}}}

\setcounter{secnumdepth}{0}
\begin{document}

\section{2. domácí úlohy}

\subsection{1a)}
\subsubsection*{Zadání}
Pro každé přirozené $\delta\geq1$ dokažte, že souvislý graf
$G=(V,E)$ s $\delta(G) = \delta$ obsahuje cestu s $\min\left\{2\delta, |V| - 1\right\}$
hranami.

\subsubsection*{Řešení}

\subsection{1b)}
\subsubsection*{Zadání}
Pro každé přirozené $\delta\geq1$ zkonstruujte nekonečně velkou
množinu grafů $\mathcal{G}_\delta$ takovou, že každý $G\in\mathcal{G}_\delta$ je souvislý,
má $\delta(G) = \delta$ a zároveň $G$ neobsahuje cestu s $2\delta +1$ hranami.

\subsubsection*{Řešení}

\subsection{2a)}
\subsubsection*{Zadání}
Pro každé $n$ a $k\leq n$ určete (a následně dokažte!) minimální počet hran grafu
s $n$ vrcholy a $k$ komponentami souvislosti.

\subsubsection*{Řešení}

\subsection{2b)}
\subsubsection*{Zadání}
Pro každé $n$ a $k\leq n$ určete (a následně dokažte!) maximální počet hran grafu
s $n$ vrcholy a $k$ komponentami souvislosti.


\subsubsection*{Řešení}

\subsection{3a)}
\subsubsection*{Zadání}
Nechť $T=(V,E)$ je strom obsahující vrchol stupně $k\geq 3$.
Dokažte, že $T$ má alespoň $k$ listů.
\subsubsection*{Řešení}

Označme vrchol stupně $k$ jako $v_0$. Vytvořme zakořeněný strom (tj. orientovaný graf) tak, že vrchol $v_0$ bude kořenem. 
Orientace stran je určena tak, aby hrany vedly směrem od kořene. \\
Definujme potomka vrcholu $a$ jako vrchol do kterého vede hrana z vrcholu $a$.

Dokažme pomocné tvrzení
\begin{lemma*}
    Z každého vrcholu v orientovaném stromu existuje cesta která je zakončená listem.
\end{lemma*}
\begin{proof}
    Existují 2 možnosti. Buď je vrchol listem (v tom případě je cesta triviální), 
    nebo zvolíme libovolný z jeho potomků a strom postupně procházíme.
    Protože má strom konečný počet vrcholů a neobsahuje cyklus, 
    vždy dorazíme do vrcholu který potomek nemá, tedy listu. 
\end{proof}

Vrchol $v_0$ je stupně $k$, má tedy $k$ potomků. Na každý z těchto vrcholů aplikujeme pomocné tvrzení.
Je tedy zřejmé, že strom má alespoň $k$ listů. \qed


\subsection{3b)}
\subsubsection*{Zadání}
Buď $G=(V,E)$ graf. Dokažte, že následující tvrzení jsou ekvivalentní.
\begin{enumerate}
    \item[(a)] $G$ je strom
    \item[(b)] $G$ neobsahuje kružnici a $|V| = |E| + 1$
\end{enumerate}

\subsubsection*{Řešení}

Z (a) dokažme (b):

Kružnici neobsahuje z definice.\\
Vezměme libovolný list stromu a odeberme ho (i s hranou která do něj vede).
Ze stromu jsme tedy odebrali jednu hranu a jeden vrchol. Graf který vznikl,
je stále stromem, protože odebáním listu graf nepřestal být souvislý a odebráním hrany jsme nemohli vytvořit cyklus.\\
Toto můžeme tedy opakovat. Nakonec se dostaneme do situace kdy nám v grafu zůstane pouze jeden vrchol.\\
Jak jsme zmínili i toto je strom, a pro něj platí rovnice $|V|=|E|+1$ triviálně.
Protože jsme se do tohoto stavu dostali postupným odečítáním 1 z obou stran rovnice, tak vztah musí platit i pro původní strom.


Z (b) dokažme (a):

Chceme ukázat že $\forall x,y \in V, \exists_1$ cesta v $G$ z $x$ do $y$. 

Potřebujeme tedy ukázat, že mezi všemi vrcholy existuje nějaká cesta (tj. graf je souvislý) a poté, že existuje pouze jedna.


\begin{lemma*}
    Pokud $G$ neobsahuje kružnici a $|V| = |E| + 1$, tak potom je $G$ souvislý.
\end{lemma*}
\begin{proof}
Pro spor předpokladáme, že není souvislý. \\
Existují tedy alespoň 2 komponenty souvislosti.\\
Graf označme jako $G = (G_V, G_E) = \bigcup_{i=1}^k K_i, K_i = (V_i, E_i)$, kde $K_i$ jsou jednotlivé komponenty souvislosti.

Navíc platí, že souvislý graf o $n$ vrcholech, má nejméně $n - 1$ hran, jak bylo ukázáno v úkolu 2a) \\
Z předpokladu platí \begin{align*}
    |G_V| &= |G_E| + 1\\
    \sum_{i} |V_i| &= \sum_i (|E_i|) + 1 > \sum_i (|V_i| - 1) + 1  
\end{align*}

Kde nerovnost platí pro $i>1$, tj. pro každý nesouvislý graf.\\
Z toho plyne, že alespoň jedna komponenta souvislosti má více než $n-1$ hran, označme ji jako $H$.
Navíc z definice platí, že komponenta souvislosti je souvislá.

Z toho, ale plyne, že $H$ není strom, protože $H$ není do inkluze vůči hranám minimální souvislý graf.\\
Nakonec využijme definice stromu, která říká, že strom je souvislý graf bez cyklů.
Protože $H$ není strom, ale je souvislý, tak musí obsahovat cyklus, což je spor.
\end{proof}


Z pomocného tvrzení tedy plyne, že mezi každými 2 vrcholy tedy existuje nějaká cesta. To, že existuje právě jedna plyne z předpokladu o neexistenci kružnici v grafu.
Pokud by totiž v grafu existovali 2 různé cesty z vrcholu $a$ do $b$, šlo by z nich triviálně vytvořit cyklus. \\Dokážeme sporem.\\
Stačilo by vzít 2 vrcholy s nejmenší vzdáleností pro které existují alespoň 2 různé cesty, cyklus by šel pak vytvořit tím, že nejprve z bodu $a$ půjdeme do bodu $b$ jednou cestou a pak z bodu $b$ do bodu $a$.
Tyto 2 cesty by měli společné pouze vrcholy $a$ a $b$ (jinak by existovaly 2 vrcholy s kratší vzdáleností) a tvořili by cyklus, což je spor s předpokladem.



Tímto jsme dokázali ekvivalenci tvrzení.
\qed

\subsection{4)}
\subsubsection*{Zadání}

Dokažte, že pro posloupnost $n\geq 2$ celých čísel 
$1\geq d_1\geq d_2 \geq \dots \geq d_n$ jsou následující
dvě podmínky ekvivalentní:
\begin{enumerate}
    \item[(a)] Existuje strom, který má skóre $(d_1, d_2, \dots, d_n)$
    \item[(b)] $\sum_{i=1}^n d_i = 2n - 2$
\end{enumerate}


\subsubsection*{Řešení}


\subsection{5)}
\subsubsection*{Zadání}

Nechť $K_n^-$ je (až na izomorfismus jednoznačně určený) graf s $n$ vrcholy a $\binom{n}{2} -1$ hranami.
Určete počet koster $K_n^-$ pro každé $n\geq 3$.

\subsubsection*{Řešení}


Nejprve najděme počet koster, které obsahují jednu konkrétní hranu. 
Provedeme to tak, že spočteme celkový počet hran všech koster úplného grafu s $n$ vrcholy dvěma různými způsoby. 

Z Caleyho formule víme, že na úplném grafu existuje $n^{n-2}$ koster. Dále víme, že každá kostra úplného grafu má $n-1$ hran, 
(plyne z definice stromu jako do inkluze vůči hranám minimálního a souvislého grafu a z cvičení 2) ), proto celkem dostáváme $n^{n-2}(n-1)$ hran.

Dále víme, že na úplném grafu patří každá hrana do stejného počtu koster 
(plyne ze symetrie úlohy, není možné aby se jedna hrana lišila), toto číslo označme
jako $k$. Dále víme, že v úplném grafu máme $\binom{n}{2}$ hran. Tedy každá hrana patří do $k$ stromů a celkem tedy máme $\binom{n}{2}k$ hran.

Tyto dva vztahy se ale zřejmě musí rovnat. Dostáváme tedy 
\begin{equation*}
    n^{n-2}(n-1) = \binom{n}{2}k
\end{equation*}
Upravme a dostaneme 
\begin{equation*}
    k = n^{n-2}\frac{(n-1)}{\binom{n}{2}} = n^{n-2}\frac{2}{n} = 2n^{n-3}
\end{equation*}
Toto je tedy počet koster úplného grafu, které obsahují jednu konkrétní hranu.


Abychom dostali počet koster $K_n^-$ stačí tedy toto číslo odečíst od celkového počtu koster. 
\begin{equation*}
    n^{n-2} - 2n^{n-3} = (n-2) n^{n-2}
\end{equation*}

Celkově tedy počet koster pro $K_n^-$ je $(n-2) n^{n-2}$.

\end{document}