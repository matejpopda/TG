\documentclass[../main.tex]{subfiles}
\graphicspath{{\subfix{../images/}}}

\setcounter{secnumdepth}{0}
\begin{document}

\section{2. domácí úlohy}

\subsection{1a)}
\subsubsection*{Zadání}
Pro každé přirozené $\delta\geq1$ dokažte, že souvislý graf
$G=(V,E)$ s $\delta(G) = \delta$ obsahuje cestu s $\min\left\{2\delta, |V| - 1\right\}$
hranami.

\subsubsection*{Řešení}

Zvolme nejdelší cestu v grafu $G$, tj. posloupnost $v_0, e_1, v_1, ..., e_n, v_n$
kde $n$ je délka cesty. Množinu všech vrcholů cesty označme jako $P$.

Platí následující 2 pozorování:

\begin{lemma*}
    Sousedé vrcholů $v_0$ a $v_n$ musí být v množině $P$, tj.\\ $N_G(v_0) \subseteq P \wedge N_G(v_n) \subseteq P$
\end{lemma*}

\begin{proof}
    Pokud by tomu tak nebylo, tak by $P$ nebyla nejdelší cesta v $G$, 
    stačilo by totiž do cesty přidat bod který je v sousedství jednoho z koncových vrcholů a ještě v cestě není.
    Což je spor s tím, že $P$ je nejdelší cesta.
\end{proof}


\begin{lemma*}
    Pro žádné $i\in\{  1,2,...,n-2 \}$ neplatí, že $v_i \in N_G(v_n) \wedge v_{i+1} \in N_G(v_0)$ nebo $P = V$.
\end{lemma*}
\begin{proof}
    Sporem, tj $\exists i, v_i \in N_G(v_n) \wedge v_{i+1} \in N_G(v_0)$.

    Pokud by tomu tak bylo, šlo by z cesty vytvořit cyklus. Jeho sled by pak byl  
    \begin{equation*}
        v_{i+1}, e_{i+2}, v_{i+2}, ..., v_n, \{v_n, v_i\}, v_i, e_i, v_{i-1}, ..., e_{1}, v_0, \{v_0, v_{i+1}\}, v_{i+1}
    \end{equation*}

    Protože ale víme, že $G$ je souvislý, někde v cestě musí existovat vrchol, 
    který má souseda mimo množinu $P$ (jedinou výjimkou je pokud $P = V$), tento vrchol označme $v_k$.
    
    Můžeme pak vytvořit cestu, která je delší než $P$, stačí začít v sousedu $v_k$ a pak projít cyklus.
    
    To je ale spor s tím, že $P$ je nejdelší cesta. 
\end{proof}

Nejprve vyřešme případ pro $2\delta < |V|$:\\
Z předchozích 2 pozorování tedy víme, že posloupnost vrcholů které tvoří cestu musí být alespoň následující.

Cesta začíná ve $v_0$ poté následuje $\delta - 1$ sousedů $v_0$,pak společný soused $v_0$ a $v_n$,
pak $\delta - 1$ sousedů $v_n$ a konečně $v_n$.

Tato cesta má celkem $2\delta +1$ vrcholů, jde tedy o cestu délky $2\delta$.

Případ $2\delta \geq |V|$ plyne z druhé části druhého tvrzení. $v_0$ a $v_n$ mají $\delta$ sousedů, 
proto mají alespoň 2 společné vrcholy, z toho neplatí první část druhého tvrzení a proto platí druhá, tedy $P = V$

Máme tedy cestu s $|V|$ vrcholy, ta má délku $|V|-1$.

Spojením těchto dvou případů dostaneme námi hledané tvrzení. 
\qed



\subsection{1b)}
\subsubsection*{Zadání}
Pro každé přirozené $\delta\geq1$ zkonstruujte nekonečně velkou
množinu grafů $\mathcal{G}_\delta$ takovou, že každý $G\in\mathcal{G}_\delta$ je souvislý,
má $\delta(G) = \delta$ a zároveň $G$ neobsahuje cestu s $2\delta +1$ hranami.

\subsubsection*{Řešení}

Zvolme libovolné pevné $\delta$.

Vytvořme následující množinu grafů $\{ G_2, G_3, ... \}$.

$G_i$ vytvořme následovně. 

Vezměme $i$ úplných grafů s $\delta$ vrcholy, označme $H_k$, mezi těmito grafy zatím neexistuje cesta.

Do grafu pak přidejme vrchol $v_0$, tento vrchol bude propojený s každým jiným vrcholem.

Je zřejmé, že indukované podgrafy $G_i$ které mají vrcholy $V(H_k) + v_0$ jsou úplné grafy s $\delta +1$ vrcholy, mají tedy stupeň $\delta$.

Z toho plyne, že pro celkový graf platí $\delta(G_i) = \delta$. Tento graf je také zřejmě souvislý. 
Navíc platí, že každá cesta mezi 2 podgrafy $H_k$ musí procházet přes $v_0$.

Z tohoto plyne, že každá cesta navštíví nejvíce 2 podgrafy $H_k$ a 
z toho dostáváme, že nejdelší možná cesta má délku $2\delta$.

V podgrafu $H_k$ totiž nalezneme cestu délky $\delta -1$, pak se cestou přes 2 hrany dostaneme do jiného podgrafu $H_j$, kde nalezneme cestu o nejvíce $\delta-1$ vrcholech.

Celkem tedy jako délku nejdelší cesty máme $2( \delta - 1) + 2 = 2\delta$. 



\subsection{2a)}
\subsubsection*{Zadání}
Pro každé $n$ a $k\leq n$ určete (a následně dokažte!) minimální počet hran grafu
s $n$ vrcholy a $k$ komponentami souvislosti.

\subsubsection*{Řešení}

Zvolme libovolné pevné $n$. 

Nejprve řešme případ s jednou komponentou souvislosti, tedy $k=1$:

Hledáme tedy souvislý graf s minimálním počtem vrcholů, to je ale z definice strom. Pro něj platí $|E| = n-1$

Případ $k>1$: 

Vezměme $G= (V,E) =( \bigcup_{i=1}^k V_i  ,\bigcup_{i=1}^k E_i )$, kde $G_i = (V_i, E_i)$ jsou různé komponenty souvislosti $G$. Navíc $n= |V|$

Platí, že $|E| = \sum_{i=1}^k|E_i|$. Protože $|E_i|>0$ stačí minimalizovat jednotlivé $E_i$.

Protože ale platí, že jednotlivé komponenty souvislosti jsou souvislé podgrafy, tak z případu pro $k=1$ víme, že to nastává pro $|E_i| = |V_i| -1$.

Celkem tedy dostáváme 
\begin{equation*}
    |E| = \sum_{i=1}^k|E_i| = \sum_{i=1}^k(|V_i| - 1) = |V| - k = n-k
\end{equation*}
\qed 


\subsection{2b)}
\subsubsection*{Zadání}
Pro každé $n$ a $k\leq n$ určete (a následně dokažte!) maximální počet hran grafu
s $n$ vrcholy a $k$ komponentami souvislosti.


\subsubsection*{Řešení}

Zvolme libovolné pevné $n$.

Definujme graf  $G= (V,E) =( \bigcup_{i=1}^k V_i  ,\bigcup_{i=1}^k E_i )$, kde $G_i = (V_i, E_i)$ jsou různé komponenty souvislosti $G$.Navíc $n= |V|$


Nejprve řešme případ s jednou komponentou souvislosti, tedy $k=1$:

Nejvíce hran má úplný graf $K_n$, pro který platí $E(K_n) = \binom{|V|}{2} = \frac{(n - 1)n}{2}$.

Případ $k=2$:

Hledáme maximum $|E| = |E_1| + |E_2|$, navíc platí, že každá komponenta musí být úplný graf (jinak bychom do ní mohli přidat hranu).

Platí, že $|V_1| =  n -|V_2|$

Platí tedy 
\begin{equation*}
    |E| = |E_1| + |E_2| = \binom{|V_1|}{2} + \binom{|V_2|}{2} =  \binom{n -|V_2|}{2} + \binom{|V_2|}{2} = 
    (|V_2| - \frac{1}{2}n)^2 + \frac{n^2}{4} -\frac{n}{2}
\end{equation*}

Výraz maximalizujeme pro $|V_2|$, $n$ je pevné, navíc platí $0 < |V_2| <n$.

Jediné podezřelé body jsou pro $|V_2| = 1, n-1, \lceil\frac{n}{2}\rceil, \lfloor\frac{n}{2}\rfloor$, jednoduchým dosazením zjišťujeme, že maxima dosahujeme pro $|V_2| = 1$ a $|V_2| = n - 1$

$G$ tedy má maximum hran, pokud je jeden vrchol osamocený, a druhá komponenta souvislosti je úplný graf.   Máme tedy
\begin{equation*}
    |E| = \binom{n - 1}{2}
\end{equation*}

Případ $k>2$:

Hledáme maximum $|E| = \sum_{i=1}^k|E_i|$.

Musí platit, že každé 2 různé komponenty souvislosti $G_j$, $G_k$ maximalizují $|E_j| + |E_k|$.
Pokud by tomu tak nebylo, mohli bychom vzít vrcholy $V_j$ a $V_k$ a vytvořit z nich 2 nové komponenty, které by měli více hran.

Z toho plyne, že mezi každou dvojicí různých komponentů souvislosti, je jeden z nich osamocený vrchol a jeden z nich úplný graf
(osamocený vrchol je také úplný graf).

Z této podmínky plyne, že máme $k-1$ osamocených vrcholů, a jeden úplný graf s $n - k + 1$ vrcholy.

Celkově tedy dostáváme 
\begin{equation*}
    |E| = \binom{n - k + 1}{2}
\end{equation*}


\subsection{3a)}
\subsubsection*{Zadání}
Nechť $T=(V,E)$ je strom obsahující vrchol stupně $k\geq 3$.
Dokažte, že $T$ má alespoň $k$ listů.
\subsubsection*{Řešení}
Označme vrchol stupně $k$ jako $v_0$. Vytvořme zakořeněný strom (tj. orientovaný graf) tak, že vrchol $v_0$ bude kořenem. 
Orientace stran je určena tak, aby hrany vedly směrem od kořene. \\
Definujme potomka vrcholu $a$ jako vrchol do kterého vede hrana z vrcholu $a$.

Dokažme pomocné tvrzení
\begin{lemma*}
    Z každého vrcholu v orientovaném stromu existuje cesta která je zakončená listem.
\end{lemma*}
\begin{proof}
    Existují 2 možnosti. Buď je vrchol listem (v tom případě je cesta triviální), 
    nebo zvolíme libovolný z jeho potomků a strom postupně procházíme.
    Protože má strom konečný počet vrcholů a neobsahuje cyklus, 
    vždy dorazíme do vrcholu který potomek nemá, tedy listu. 
\end{proof}

Vrchol $v_0$ je stupně $k$, má tedy $k$ potomků. Na každý z těchto vrcholů aplikujeme pomocné tvrzení.
Je tedy zřejmé, že strom má alespoň $k$ listů. \qed


\subsection{3b)}
\subsubsection*{Zadání}
Buď $G=(V,E)$ graf. Dokažte, že následující tvrzení jsou ekvivalentní.
\begin{enumerate}
    \item[(a)] $G$ je strom
    \item[(b)] $G$ neobsahuje kružnici a $|V| = |E| + 1$
\end{enumerate}

\subsubsection*{Řešení}

Z (a) dokažme (b):

Kružnici neobsahuje z definice.\\
Vezměme libovolný list stromu a odeberme ho (i s hranou která do něj vede).
Ze stromu jsme tedy odebrali jednu hranu a jeden vrchol. Graf který vznikl,
je stále stromem, protože odebáním listu graf nepřestal být souvislý a odebráním hrany jsme nemohli vytvořit cyklus.\\
Toto můžeme tedy opakovat. Nakonec se dostaneme do situace kdy nám v grafu zůstane pouze jeden vrchol.\\
Jak jsme zmínili i toto je strom, a pro něj platí rovnice $|V|=|E|+1$ triviálně.
Protože jsme se do tohoto stavu dostali postupným odečítáním 1 z obou stran rovnice, tak vztah musí platit i pro původní strom.


Z (b) dokažme (a):

Chceme ukázat že $\forall x,y \in V, \exists_1$ cesta v $G$ z $x$ do $y$. 

Potřebujeme tedy ukázat, že mezi všemi vrcholy existuje nějaká cesta (tj. graf je souvislý) a poté, že existuje pouze jedna.


\begin{lemma*}
    Pokud $G$ neobsahuje kružnici a $|V| = |E| + 1$, tak potom je $G$ souvislý.
\end{lemma*}
\begin{proof}
Pro spor předpokladáme, že není souvislý. \\
Existují tedy alespoň 2 komponenty souvislosti.\\
Graf označme jako $G = (G_V, G_E) = \bigcup_{i=1}^k K_i, K_i = (V_i, E_i)$, kde $K_i$ jsou jednotlivé komponenty souvislosti.

Navíc platí, že souvislý graf o $n$ vrcholech, má nejméně $n - 1$ hran, jak bylo ukázáno v úkolu 2a) \\
Z předpokladu platí \begin{align*}
    |G_V| &= |G_E| + 1\\
    \sum_{i} |V_i| &= \sum_i (|E_i|) + 1 > \sum_i (|V_i| - 1) + 1  
\end{align*}

Kde nerovnost platí pro $i>1$, tj. pro každý nesouvislý graf.\\
Z toho plyne, že alespoň jedna komponenta souvislosti má více než $n-1$ hran, označme ji jako $H$.
Navíc z definice platí, že komponenta souvislosti je souvislá.

Z toho, ale plyne, že $H$ není strom, protože $H$ není do inkluze vůči hranám minimální souvislý graf.\\
Nakonec využijme definice stromu, která říká, že strom je souvislý graf bez cyklů.
Protože $H$ není strom, ale je souvislý, tak musí obsahovat cyklus, což je spor.
\end{proof}


Z pomocného tvrzení tedy plyne, že mezi každými 2 vrcholy tedy existuje nějaká cesta. To, že existuje právě jedna plyne z předpokladu o neexistenci kružnici v grafu.
Pokud by totiž v grafu existovali 2 různé cesty z vrcholu $a$ do $b$, šlo by z nich triviálně vytvořit cyklus. \\Dokážeme sporem.\\
Stačilo by vzít 2 vrcholy s nejmenší vzdáleností pro které existují alespoň 2 různé cesty, cyklus by šel pak vytvořit tím, že nejprve z bodu $a$ půjdeme do bodu $b$ jednou cestou a pak z bodu $b$ do bodu $a$.
Tyto 2 cesty by měli společné pouze vrcholy $a$ a $b$ (jinak by existovaly 2 vrcholy s kratší vzdáleností) a tvořili by cyklus, což je spor s předpokladem.



Tímto jsme dokázali ekvivalenci tvrzení.
\qed

\subsection{4)}
\subsubsection*{Zadání}

Dokažte, že pro posloupnost $n\geq 2$ celých čísel 
$1\leq d_1\leq d_2 \leq \dots \leq d_n$ jsou následující
dvě podmínky ekvivalentní:
\begin{enumerate}
    \item[(a)] Existuje strom, který má skóre $(d_1, d_2, \dots, d_n)$
    \item[(b)] $\sum_{i=1}^n d_i = 2n - 2$
\end{enumerate}


\subsubsection*{Řešení}


Z (a) dokažme (b):

Nechť existuje strom $T = (V,E)$, který má skóre $(d_1, d_2, \dots, d_n)$. $T$ je tedy souvislý a $|E|=|V|-1$.

Potom platí (z handshaking lemmatu)
\begin{equation*}
    \sum_{i=1}^n d_i = \sum_{v\in V}deg(v) = 2|E| = 2(|V| - 1) = 2(n-1)
\end{equation*}

Z (b) dokažme (a):

Dokážeme indukcí.

Pro $n=2$:
$1\leq d_1\leq d_2$, $d_1 + d_2 = 2$. Z toho máme $d_1 = d_2 = 1$.

Jako strom zvolme $T=([2], \{ 1,2\})$. Tento strom má zřejmě skóre $(1,1)$.

Indukční krok:


Indukční předpoklad tedy je, že tvrzení platí pro $n$, tj. pokud $\sum_{i=1}^n d_i = 2n - 2$ tak existuje strom který má skóre $(d_1, d_2, \dots, d_n)$.

Tvrzení dokažme pro $n+1$.

Víme, že 
\begin{equation*}
    \sum_{i=1}^{n+1} d_i = 2(n+1) - 2
\end{equation*}

kde $1\leq d_1\leq d_2 \leq \dots \leq d_{n+1}$.

Z tohoto plyne, že $d_1 =1$ (jinak by suma byla větší než pravá strana) a $d_{n+1} \geq 2$ (jinak by suma byla menší než pravá strana).

Z posloupnosti $(d_1, d_2, ..., d_{n+1})$ odeberme $d_1$ a $d_{n+1}$ zaměňme za $d_{n+1} -1$.

Z této posloupnosti, vytvoříme neklesající posloupnost čísel $g_1, ..., g_n$, ($g_n$ nemusí nutně být $d_{n+1}$) pro kterou platí 
\begin{equation*}
    \sum_{i=1}^{g+1} d_i = 2(n+1) - 2
\end{equation*}

Podle indukčního předpokladu pro tuto posloupnost existuje strom $T_n$ s jejím skóre.

Tento strom obsahuje $v\in V$, $deg(v) = d_{n+1} - 1$, k tomuto vrcholu připojíme list, 
a tím se změní stupeň $v$ na, $deg(v) = d_{n+1}$, zároveň má list stupeň 1, dáme ho na začátek posloupnosti. Tím dostáváme strom se skóre
$(d_1, d_2, ..., d_n)$.

Nakonec doplníme, že tento nový graf je stále stromem, neboť přidáním listu nemůže vzniknout cyklus nebo vzniknout nová komponenta souvislosti.



\subsection{5)}
\subsubsection*{Zadání}

Nechť $K_n^-$ je (až na izomorfismus jednoznačně určený) graf s $n$ vrcholy a $\binom{n}{2} -1$ hranami.
Určete počet koster $K_n^-$ pro každé $n\geq 3$.

\subsubsection*{Řešení}


Nejprve najděme počet koster, které obsahují jednu konkrétní hranu. 
Provedeme to tak, že spočteme celkový počet hran všech koster úplného grafu s $n$ vrcholy dvěma různými způsoby. 

Z Caleyho formule víme, že na úplném grafu existuje $n^{n-2}$ koster. Dále víme, že každá kostra úplného grafu má $n-1$ hran, 
(plyne z definice stromu jako do inkluze vůči hranám minimálního a souvislého grafu a z cvičení 2) ), proto celkem dostáváme $n^{n-2}(n-1)$ hran.

Dále víme, že na úplném grafu patří každá hrana do stejného počtu koster 
(plyne ze symetrie úlohy, není možné aby se jedna hrana lišila), toto číslo označme
jako $k$. Dále víme, že v úplném grafu máme $\binom{n}{2}$ hran. Tedy každá hrana patří do $k$ stromů a celkem tedy máme $\binom{n}{2}k$ hran.

Tyto dva vztahy se ale zřejmě musí rovnat. Dostáváme tedy 
\begin{equation*}
    n^{n-2}(n-1) = \binom{n}{2}k
\end{equation*}
Upravme a dostaneme 
\begin{equation*}
    k = n^{n-2}\frac{(n-1)}{\binom{n}{2}} = n^{n-2}\frac{2}{n} = 2n^{n-3}
\end{equation*}
Toto je tedy počet koster úplného grafu, které obsahují jednu konkrétní hranu.


Abychom dostali počet koster $K_n^-$ stačí tedy toto číslo odečíst od celkového počtu koster. 
\begin{equation*}
    n^{n-2} - 2n^{n-3} = (n-2) n^{n-2}
\end{equation*}

Celkově tedy počet koster pro $K_n^-$ je $(n-2) n^{n-2}$.

\end{document}