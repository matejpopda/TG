\documentclass[../main.tex]{subfiles}
\graphicspath{{\subfix{../images/}}}

\setcounter{secnumdepth}{0}
\begin{document}

\section{Těžká úloha 3}

\subsection{Zadání a)}
Dokažte, že počet neizomorfních stromů s $n$ vrcholy je nejvýše $4^n$.

\subsection{Řešení a)}




\subsection{Zadání b)}
Dokažte, že $G=(V,E)$ splňující $|V|\geq 3$ a $|E| < \binom{|V|}{2}$ je strom $\Leftrightarrow$ pro každé $f\in\binom{V}{2}\setminus E$ platí, že $G + f$ obsahuje právě jeden cyklus.
\subsection{Řešení b)}

Dokažme $(\implies)$: 

Definice stromu je graf který jej do inkluze vůči hranám maximální graf neobsahující cyklus. Přidáním vrcholu tedy cyklus vznikne.

Ukažme sporem, že přidáním hrany vznikne právě jeden cyklus. 

Předpokládejme tedy, že přidáním hrany do stromu vznikly alespoň 2 cykly. 

Dokažme $(\impliedby)$:

Přidáním hrany do grafu se počet cyklů může pouze zvýšit.
Protože přidáním hrany vznikne právě jeden cyklus, tak původní graf může mít buď 1 nebo 0 cyklů.

Pokuď měl původní graf 0 cyklů, tak byl vůči hranám maximální bez cyklů, tedy strom.

Nakonec sporem ukažme, že původní graf nemůže obsahovat cyklus.

Předpokládejme tedy, že v grafu $G=(V,E)$ existuje právě jeden cyklus a dále $\forall f\in \binom{V}{2}\setminus E$ platí, že $G+f$ obsahuje právě jeden cyklus.

Pokud by délka cyklu byla větší než 4, tak bychom mohli přidat novou hranu, která spojuje dva vrcholy z cyklu, tím by vznikly 2 další cykly, což by byl spor.

Mějme tedy cyklus o délce 3. 

\end{document}